\DeclareMathOperator*{\argmax}{arg\,max}
\DeclareMathOperator*{\argmin}{arg\,min}
\setlength{\parindent}{0em}
\setlength{\parskip}{1em}
\renewcommand{\baselinestretch}{1}
\newenvironment{thisnote}{\par\color{blue}}{\par}
\makeatletter
\def\FN@baselinestretch{.6}
\makeatother
\renewcommand\qedsymbol{$\blacksquare$}
\newenvironment{solution}
 {\renewcommand\qedsymbol{$\blacksquare$}\begin{proof}[Solution]}
 {\end{proof}}
\newcommand{\dd}{\text{ d}}
\newcommand{\E}[1]{\mathbb{E}\left[#1\right]}
\newcommand{\Ex}[2]{\mathbb{E}_{#1}\left[#2\right]}
\newcommand{\Var}[1]{\text{Var}\left(#1\right)}
\newcommand{\Cov}[1]{\text{Cov}\left(#1\right)}
\newcommand{\pr}[1]{\text{Pr}\left(#1\right)}
\newcommand{\PP}[1]{\mathbb{P}\left[#1\right]}
\newcommand{\pto}{\overset{p}{\to}}
\newcommand{\dto}{\overset{d}{\to}}
\newcommand{\Perp}{\perp\!\!\!\perp}

\bibliographystyle{apalike}


\newcommand{\foo}{\makebox[0pt]{\textbullet}\hskip-0.5pt\vrule width 1pt\hspace{\labelsep}}

\def\citeapa #1{\citeauthor{#1}, \citeyear{#1}}

\hypersetup{
    colorlinks,
    linkcolor={red!100!black},
    citecolor={blue!100!black},
    urlcolor={blue!100!black}
}

\theoremstyle{plain}
\newtheorem{thm}{Theorem}[subsection]
\newtheorem{lem}[thm]{Lemma}
\newtheorem{prop}[thm]{Proposition}
\newtheorem*{cor}{Corollary}

\theoremstyle{definition}
\newtheorem{defn}{Definition}[subsection]
\newtheorem{conj}{Conjecture}[section]
\newtheorem{exmp}{Example}[subsection]

\theoremstyle{remark}
\newtheorem*{rem}{Remark}
\newtheorem*{note}{Note}

\DeclareCoupledCountersGroup{thms}
\DeclareCoupledCounters[name=thms]{thm, defn, exmp}

% For tables - From Kenneth's Automata and Bayesian Updating
\tikzstyle{rect}=[rectangle, rounded corners, fill=white!20, text centered, draw=black,  minimum height=1.7em]
\tikzstyle{rectb}=[rectangle, rounded corners, fill=white!20, text centered, draw=blue,  minimum height=1.7em]

% Two node styles: solid and hollow - for game trees
\tikzstyle{solid node}=[circle,draw,inner sep=1.2,fill=black]
\tikzstyle{hollow node}=[circle,draw,inner sep=1.2]

% From overleaf template

% MATH SHORTHANDS
% Partial Differential of #1 w.r.t. #2
\newcommand{\dpartial}[2]{\frac{\partial #1}{\partial #2}}

% \ceil{x} instead of \lceil x \rceil
% Same for \floor{x}
\DeclarePairedDelimiter\ceil{\lceil}{\rceil}
\DeclarePairedDelimiter\floor{\lfloor}{\rfloor}
\newcommand{\abs}[1]{\left| #1 \right|}

% Auto-resize () and []
\newcommand*\autoop{\left(}
\newcommand*\autocp{\right)}
\newcommand*\autoob{\left[}
\newcommand*\autocb{\right]}
\AtBeginDocument {%
   \mathcode`( 32768
   \mathcode`) 32768
   \mathcode`[ 32768
   \mathcode`] 32768
   \begingroup
       \lccode`\~`(
       \lowercase{%
   \endgroup
       \let~\autoop
   }\begingroup
       \lccode`\~`)
       \lowercase{%
   \endgroup
       \let~\autocp
   }\begingroup
       \lccode`\~`[
       \lowercase{%
   \endgroup
       \let~\autoob
   }\begingroup
       \lccode`\~`]
       \lowercase{%
   \endgroup
       \let~\autocb
}}
\delimiterfactor 1001
\makeatletter
\AtBeginDocument {%
          \def\resetMathstrut@{%
           \setbox\z@\hbox{\the\textfont\symoperators\char40}%
           \ht\Mathstrutbox@\ht\z@ \dp\Mathstrutbox@\dp\z@}%
}%
\makeatother

\newcommand{\vecbr}[1]{\langle #1 \rangle}

% Unit vectors

\newcommand{\ui}{\hat{\imath}}
\newcommand{\uj}{\hat{\jmath}}
\newcommand{\uk}{\hat{k}}
\newcommand{\V}{\vec{V}}

% Common expressions

\newcommand{\half}[1]{\frac{#1}{2}}
\newcommand{\recip}[1]{\frac{1}{#1}}
\newcommand{\invsqrt}[1]{\recip{\sqrt{#1}}}
\newcommand{\halfpi}{\half{\pi}}

% Integral evaluation bar
\newcommand{\windbar}[2]{\Big|_{#1}^{#2}}
\newcommand{\rightinfwindbar}[0]{\Big|_{0}^\infty}
\newcommand{\leftinfwindbar}[0]{\Big|_{-\infty}^0}

% Column type "L" for tabular environment, contents of column are in display mode by default
\newcolumntype{L}{>{$}l<{$}}

% Circled single character (used for state machine notation)

\newcommand{\state}[1]{\large\protect\textcircled{\textbf{\small#1}}}

% FORMATTING
% Remove section numbers
\makeatletter
\renewcommand{\@seccntformat}[1]{}
\makeatother
% Indent text within section
% \leftskip=2em
% Center section headings
\titleformat{\section}[block]{\Large\bfseries\filcenter}{}{1em}{}
% `enumerate` environment uses (\alph) format
\setlist[enumerate]{label=(\alph*)}

\newcommand{\shrule}{\\ \centerline{\rule{13cm}{0.4pt}}}